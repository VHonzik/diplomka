This thesis analysed a dataset of human to human interaction in a shared navigation task. Focusing on referring expressions, it revealed a new referencing strategy, which is not following the methodology of previous research in REG field. 

After describing the shared task and the dataset, this thesis attempted to model the referencing strategy and other related NLG problems using machine learning techniques. Spatial features, such as information about the scene complexity, were successfully used in related research and were therefore extracted from the dataset. They, however, proved to have less predictive power in this task. This paper argues that the complexity of the shared task and dataset size had an influence on these results.

Finally, this thesis talks about a conducted experiment on the mentioned referencing strategy. The task proficiency of two NLG systems was compared in this experiment. One of the systems represented the newly discovered strategy, while the other one followed more standard methodology of REG. No significant difference between the systems' proficiency was found, suggesting the new referencing strategy neither improves nor harms system's efficiency and has more to do with human preferences and personal strategies.

It would be interesting to extract features which include more information about instruction givers and followers from the dataset and see if the attempted machine learning techniques will improve. Focusing on local and personal referencing strategies rather than global ones, is another way to continue in the work of this thesis. It might also be worth exploring a different dataset with similar complexity.

