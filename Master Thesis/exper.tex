\chapter{Experiment on RE}
Trough out this thesis, I was interested in a strategy of referencing, where the first reference does not always uniquely identifies the target object. Instead, the strategy relied on feedback and additional reference, which together with the first reference formed a chain of referenced. After not being able to model this behaviour through methods of machine learning, I decided to look at that strategy under more controlled conditions of an experiment.

\section{Hypothesis}
To evaluate this strategy I decided to compare it to the more standard approach of unique identification in the first reference.
\section{Experimental set-up}
\section{Results}