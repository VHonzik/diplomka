This thesis deals with language realization for navigating in a virtual environment. A dataset of spoken interaction from a shared navigation task is used to model the language realization. The thesis explores a particular strategy of referencing, which differs from the traditional methodology used in the relevant research. It consists of five chapters.

The first chapter describes the background of the navigation and especially of the language realization in navigation. It introduces the domain of Referring Expression Generation and a related research.

In the second chapter the GIVE framework is introduced. The whole thesis is built upon this framework and is therefore vital for following chapters. 

The third chapter analyses the spoken interaction of the S-GIVE dataset. The S-GIVE was created in GIVE framework and this thesis is one of the first papers to analyse it.

The fourth chapter attempts to create models of Natural Language Generation using machine learning techniques. Using the data from the S-GIVE dataset, it explores various sub-problems of the language realization.

In the last chapter an experiment to evaluate the mentioned referencing strategy and its results are presented. Once again built upon the GIVE framework, it compares the newly discovered strategy with more traditional methodology.   