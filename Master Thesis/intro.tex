This thesis deals with language realization for navigating in a virtual environment. It uses dataset of spoken interaction from a shared navigation task to model the language realization. It explores a particular strategy of referencing, which differs from the traditional methodology used in relevant research areas. It consists of five chapters.

The first chapter describes background for navigation and especially for language realization. It summarizes the domain of referring expression generation and related research.

In the second chapter the GIVE framework is introduced. The whole thesis is built upon this framework and is therefore vital for following chapters. 

The third chapter analyses the spoken interaction of the S-GIVE dataset. The S-GIVE was created in GIVE framework and this thesis is one of the first works to analyse it.

The fourth chapter attempts to create models of Natural Language Generation using machine learning techniques. Using data from the S-GIVE dataset, it explores various sub-problems of the language realization.

In the last chapter an experiment to evaluate referencing strategy and its results are presented. Once again built upon the GIVE framework, it compares the newly discovered strategy with more traditional methodology.   