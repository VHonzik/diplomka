This thesis deals with language generation for navigating in a virtual environment. A dataset of spoken interactions from a shared navigation task is used to model the language generation. The thesis explores a particular strategy of referring, which differs from the traditional methodology used in the relevant research. 

It consists of five chapters.

The first chapter describes the background of the navigation and especially of the language generation in navigation. It introduces the domain of Referring Expression Generation and a related research.

In the second chapter the GIVE framework is introduced. The whole thesis is built upon this framework and it is therefore vital for the following chapters. 

The third chapter introduces the spoken interaction of the S-GIVE dataset. The S-GIVE dataset was created in the GIVE framework and this thesis is one of the first papers to analyse it.

With the fourth chapter starts my contribution. The chapter analyses the S-GIVE dataset and attempts to create models of Natural Language Generation using machine learning techniques. Using the data from the S-GIVE dataset, it explores various sub-problems of language generation.

In the fifth chapter, an experiment to evaluate the mentioned referring strategy is described. Once again built upon the GIVE framework, it compares the newly discovered strategy with more traditional approaches.   

The goals of this thesis are as follows. Analyse the S-GIVE dataset. General statistics, referring expressions and the new referring strategy are the focus of the analysis. Use machine learning techniques to model the referring expression generation in the GIVE scenario. Spatial features are of a particular interest of this thesis. Compare the task efficiency of the new referring strategy with strategy inspired by current research in referring expression generation. The comparison is done through an experiment with human subjects. 