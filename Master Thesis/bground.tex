\chapter{Background}
\label{chap:bg}
Navigation is a broad problem intervening with different scientific areas. Generally speaking, a path to the destination must be found first. That problem is solved by path-planning, problem commonly associated with robotics, and specifically by pathfinding, a well-known and studied problem of artificial intelligence. Once a path is known, it must be realized into a natural language. The realization is studied by Natural Language Generation (NLG), a part of the discipline of Natural Language Processing (NLP). On top of that, one has to consider the domain the navigation is intended for. Differences in the design between car navigation and art gallery guide are immense. This paragraph was of course a very coarse description of navigation problem.

What I would like to briefly touch on is why is navigation still an interesting problem, when navigation devices are nowadays almost a necessary part of the car equipment. Car navigation is a specific navigation problem. Great availability of maps is one of the reasons why the car navigation is so developed and massively produced. It has limited sense of direction, because cars only moves on roads (as far as navigation is concerned anyway). The roads can fork and form complicated cross-roads, but that isn't still nowhere close to a free movement of, for example, walking. The limited sense of directions ties to a relatively limited vocabulary. Thanks to usually well-defined space where car moves, the navigation system's world representation can be relatively straightforward. The car navigation is still a complicated problem, but being able to relatively well navigate drivers around mapped areas tells us a little about other navigation domains.

Primary interest of this thesis is NLG part of navigation and more specifically referring expression generation, which I will introduce in the following section.  


\section{Referring expression generation}
\label{sec:bg-reg}
From the complex and comprehensive problem of navigating persons, I'm especially interested in language realization subproblem. Moreover, I limited my research mostly to discipline of referring expression generation (REG). In the dataset and framework, which I'll present in the following chapters, the REG is especially important part of language generation. In fact, in the virtual world I'm working with, extending REG module to a complete language realization system is trivial task of adding few verbs.

\citet{krahmer2012computational} created very well written survey of history and development of REG up to recent times. Following the principle of ``Don't reinvent the wheel,'' I will provide only short explanation of what REG is and brief overview of its history, both of them being heavily inspired by the survey.

REG belongs to the domain of Natural Language Generation (NLG). It is concerned with generating referring expressions (RE) to an object(s) of interest. Suppose we have three buttons next to each other and we need one of them to be pressed. Also suppose we are not able to do it ourself right now, but other person is available nearby. Most individuals would have no trouble address the person nearby and ask him to press the button. Part of their utterance would ``point out'' which button of the three, they need to be pressed. That part of utterance is a RE. Producing understandable and effective RE is for most speakers relatively easy task. But for computer programs it is not so. Context of the real world application is usually very large and if we take into account mutual relationships between entities in the context (such as a button is next to a painting), the number of possible combinations quickly grows to problematic numbers. 

First REG research appeared in the 1980s. \citet{krahmer2012computational} states that, influenced by methodologies of computation linguistic in that time, they studied REG as a part of larger speech act and doing so on hard and often anomalous cases. In 1990s, famous paper of \citet{dale1995computational} shifted the focus on determining which properties should be used, when the goal is to identify the referent, while avoiding being more informative than is required. New aim of REG was generating human-like descriptions.  According to  \citet{krahmer2012computational}, 1990s also spawned first REG algorithms in well defined REG problem, such as influential Incremental Algorithm. However, the research was limited to target being just one object, simple knowledge representation, no vague properties, all object being equally salient and ignoring the stage of surface realization of chosen properties. ``A substantial part of recent REG research is dedicated to lifting one or more of these simplifying assumptions.'' \citep{krahmer2012computational} 

Apart from lifting the assumptions, recent REG research was interested in evaluation of REG algorithms. We can also see tendency to move from simple well-defined environments, towards more natural and complex ones.

Exploring a dataset of spoken navigation through complex 3D virtual world, I have noticed some behaviours which slightly deviates from conventional focus of REG research. Speakers does not necessarily produce a reference which uniquely identifies the referent in current context. Instead, first they produce a reference which only partially identifies the referent and rely on feedback and additional RE when necessary. This thesis is primarily focused on this deviation.

\section{Related work}
\label{sec:relwork}
In this section I will briefly present work from the area of REG, which I deem relevant to my research.

\citet{ha2012combining} talk about an `information gaps' caused by existence of a non-dialogue communication stream. They concluded that the posture of user, an example of implicit information from the non-dialogue streams, is a significant attribute in modeling of dialog acts. Their goal is to overcome this `information gaps' through machine learning techniques. A shared view of the virtual world in this thesis is also a form of non-dialogue stream, with which must the navigation system work. I also try to apply machine learning to help with language generation.

\citet{viethen2011generating} compare traditional algorithmic approaches with alignment approaches based on psycho-linguistic models for the REG. They use large data-set (16,358 referring expressions) of direction giving task on a shared 2D visual scene introduced by \citet{louwerse2007multimodal}. They use three feature sets: traditional REG, alignment and independent (general information about the scene) to build decision tree models (concretely C4.5) combining these feature sets. The alignment based models outperform the traditional REG ones and the best model combines all feature sets to achieve accuracy 58.8\% and DICE score 0.81. Not using traditional algorithmic REG features did not result in a significant decrease of accuracy, suggesting that the visual context doesn't play such an important role as it was believed in the REG research so far.  \citet{viethen2011impact} verified this surprising conclusion by varying the visual context. They argue that the relative simplicity of visual scenes used in contemporary research might be the cause of insignificance of the visual context. I would argue that 3D  virtual world explored in this thesis is more complex then theirs and therefore this paper can provide some insight into these questions.  

\citet{stoia2006sentence} were interested in timing of the first reference to the target in 3D virtual world. They predicted whether direction giver refers to the target or delay the reference based on the spatial data. Their attributes were angle and distance to the target, number of visible distractors (either same category as target or all of them) and whether the target is visible. The most important feature in decision tree model was number of visible distractors followed by angle and distance. They achieved 86\% accuracy, compared 70\% baseline. The baseline was to refer when the target is visible and to delay the reference when the target isn't visible. Part of the machine learning attempts of this thesis is to replicate first reference timing of \citet{stoia2006sentence} on GIVE dataset.

\citet{stoia2006noun} developed decision trees to generate a noun phrase, specified by three slots: determiner/quantifier, pre-modifier/post-modifier and head noun. They used a data-set from 3D virtual world navigation task similar to GIVE dataset. Four categories of features were used: dialog history, spatial and visual features, relation to other objects in the world and object category. The decision trees revealed significant dependencies between the slots and importance of the spatial features. Interestingly, they used three types of system's evaluation. The exact match evaluation produced 31.2\% accuracy compared to 20\% most-frequent baseline. Comparison with hand-crafted Centering algorithm \citep{kibble2000integrated} ended with similar accuracy, favoring the machine-learning approach for requiring less structural analysis of the input text. Lastly, when human judged the system output, it was at least equal or preferred to original spontaneous language in 62.6\% (inter-annotator reliability $\kappa = 0.51$).

\citet{gallo2008production} showed that the Fruit Carts corpus can be used in NLG by case study on message complexity and structural realizations. A logistic regression confirmed that the complexity of verb arguments affects production choice between mono-clausal or bi-clausal structure. In more general terms, the complexity of the virtual environment affects how people speak on all linguistic levels. Referring expression generation should take that into consideration.

\citet{clark2004speaking} were examining speakers' monitoring of addressees in a Lego-building experiment. One participant - director - knew 10 Lego models and how to build them. The director was verbally instructing second participant - builder - to build these models. In one group the director could see the builders workspace, in second group he could not and in a third the instructions were audio-taped and simply passed onto the builder. Builders communicated with the directors on the workspace through head gestures and manipulating blocks (placing, exhibiting or poising and so on). When the workspace was blocked of, the task took much longer. In the audio-taped group the builders made many more errors. Directors often altered their utterances midcourse based on builders actions.

\citet{koller2012enhancing} tracked hearer gaze using camera and used that information to produce feedback to correct or confirm previous referring expressions. Experiment took place in a 3D virtual world. This enhancement was compared with feedback based on virtual agent's position and system with no feedback at all. Eye-tracking enhancement significantly improved hearers understanding of the REs. Eye-tracking is therefore useful tool to improve interaction quality. This experiment also shows importance of feedback, since the system with no feedback performed worse than the two systems with feedback.